\begin{frame}
    \frametitle{Introduction}
    Multiple new reactor designs require \gls{HALEU} fuel, which allows for:
    \begin{itemize}
        \item Longer cycle times
        \item Higher burnups 
    \end{itemize}
    To meet the \gls{HALEU} demand, the U.S. \gls{DOE} has proposed two methods
    \cite{griffith_overview_2020}:
    \begin{itemize}
        \item Enrichment of natural uranium
        \item Recovery and downblending of \gls{HEU}
    \end{itemize}
    Determining which method to use, or how to combine them, will depend on 
    the material requirements of the reactor(s) deployed.

\end{frame}

\begin{frame}
    \frametitle{Objectives}
    This work simulates multiple transition scenarios to \gls{HALEU}-fueled 
    reactors and aims to:
    \begin{itemize}
        \item Quantify material requirements of the transition to reactors 
              fueled by \gls{HALEU}:
              \begin{itemize}
                  \item Number of reactors deployed 
                  \item Ability to meet energy demand
                  \item Mass of uranium supplied to reactors
                  \item \gls{SWU} capacity to enrich uranium
              \end{itemize}
        \item Compare the material requirements of a small reactor with a long cycle 
              time and a medium-sized reactor with on-line refueling
        \item Identify how each \gls{HALEU} production method can be used to 
              meet the material requirements
    \end{itemize}
\end{frame}