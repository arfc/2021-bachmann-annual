\begin{frame}
    \frametitle{Introduction}
    Mutliple new reactor designs will require \gls{HALEU} fuel, whichallows for 
    \begin{itemize}
        \item Longer cycle times
        \item Higher burnups 
    \end{itemize}
    To meet the \gls{HALEU} demand, the U.S. \gls{DOE} has proposed two methods
    \cite{griffith_overview_2020}:
    \begin{itemize}
        \item Recovery and downblending of \gls{HEU}
        \item Enrichment of natural uranium
    \end{itemize}

\end{frame}

\begin{frame}
    \frametitle{Motivation}
    Each of the methods proposed by \gls{DOE} has their own advantages and 
    disadvantages \par
    Downlending of \gls{HEU}
    \begin{itemize}
        \item Does not require new material extraction
        \item Limited by \gls{HEU} supply adn dowonblending capabilities
    \end{itemize}
    Enriching natural uranium 
    \begin{itemize}
        \item Natural uranium supply is larger than \gls{HEU} supply
        \item Limited by \gls{SWU} capacity of domestic centrifuge facilities
    \end{itemize}
\end{frame}

\begin{frame}
    \frametitle{Objectives}
    This work aims to 
    \begin{itemize}
        \item Quantify material requirements of the transition to reactors 
              fueled by \gls{HALEU}
        \item Compare teh material requirements of a reactor with a long cycle 
              time and a reactor with on-line refueling
        \item Identify how each \gls{HALEU} production method can be used to 
              meet the material requirements
    \end{itemize}
\end{frame}