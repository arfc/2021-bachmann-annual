\section{Methodology}

Each fuel cycle scenario is simulated using \Cyclus, an 
agent-based fuel cycle simulator \cite{huff_fundamental_2016}. 
The agent-based architecture of \Cyclus allows for the simulation to treat
each facility discretely, including their deployment and 
decommissioning \cite{huff_fundamental_2016}. The \Cyclus 
Additional Modules Repository (the \Cycamore library) provides 
the archetypes for each of the facility agents deployed in the simulation.

The first scenario only simulates the current fleet of U.S. reactors. This 
simulation begins in January 1965 and lasts through December 2090 (125 
years in total) with one month timesteps. The \gls{IAEA} \gls{PRIS} 
database \cite{noauthor_power_1989} 2020 
Year-end Reactor Status Report provided information about each reactor.
This database provides reactor type, rated power level, start-up, and 
shutdown dates for each of the commercial reactors in the U.S. reactor 
fleet (as of December 2020). The simulations assume 
the \gls{LWR}s operate for 60 years after their
initial start date if they have not already been shut down. Only reactors
with a power level above 400 MWe were 
used in the simulation to avoid including prototype and research reactors 
present in the database. Generic reactor core masses were obtained from 
\cite{todreas_nuclear_2012} and \cite{cacuci_handbook_2010}. 

Other fuel cycle facilities deployed include: a uranium mine (to 
represent the Cameco Smith Ranch-Highland Mine), a uranium mill (to 
represent White Mesa Uranium Mill), a conversion facility (to represent 
the Honeywell Uranium Conversion Facility), an enrichment plant (to represent 
the Portsmouth Gaseous Diffusion Plant), a fuel fabrication facility (to 
represent the Westinghouse Fuel Fabrication Facility), and various waste and 
spent fuel storage and disposal facilities (to represent Yucca Mountain). 
While there
are other fuel cycle facilities in the actual U.S. fuel cycle, this 
modeling provides a concise total of the resources required 
by each non-reactor step in the nuclear fuel cycle and simplifies the 
facilities that are not expected to be deployed or decommissioned often 
during the specified time frame. Most of these facilities do not have 
any limits on their production throughputs, except the fuel fabrication 
plant, which has a limit of storing 1 million kg of enriched uranium. 
Figure \ref{fig:fuel_cycle} shows 
the facilities and material flow in the simulated fuel cycles. There are 
multiple reactor agents in each of the scenarios, one agent for each reactor, 
which are condensed to one facility in the diagram. The facilities shown in
red are used only in the transition scenarios to provide a separate 
material stream. 


\begin{figure}[ht]
    \centering
    \begin{tikzpicture}[node distance=1.5cm]
        \node (mine) [facility] {Uranium Mine};
        \node (mill) [facility, below of=mine] {Uranium Mill};
        \node (conversion) [facility, below of=mill] {Conversion};
        \node (enrichment) [facility, below of=conversion, xshift=-2cm]{Enrichment};
        \node (enrichment2) [transition, below of=conversion, xshift=2cm]{Enrichment};
        \node (fabrication2) [transition, below of=enrichment2,yshift=-1cm]{Fuel Fabrication};
        \node (fabrication) [facility, below of=enrichment, yshift=-1cm]{Fuel Fabrication};
        \node (reactor) [facility, below of=fabrication]{Reactor};
        \node (adv_reactor) [transition, below of=fabrication2]{Advanced Reactor};
        \node (wetstorage) [facility, below of=reactor]{Wet Storage};
        \node (drystorage) [facility, below of=wetstorage]{Dry Storage};
        \node (sinkhlw) [facility, below of=drystorage, xshift=2cm]{HLW Sink};
        \node (sinkllw) [facility, below of=enrichment, xshift=2cm]{LLW Sink};

        \draw [arrow] (mine) -- node[anchor=east]{Natural U} (mill); 
        \draw [arrow] (mill) -- node[anchor=east]{U$_3$O$_8$}(conversion); 
        \draw [arrow] (conversion) -- node[anchor=east]{UF$_6$}(enrichment);
        \draw [arrow] (enrichment) -- node[anchor=east]{Enriched U}(fabrication);
        \draw [arrow] (conversion) -- node[anchor=east]{UF$_6$}(enrichment2);
        \draw [arrow] (enrichment2) -- node[anchor=west]{HALEU}(fabrication2);
        \draw [arrow] (enrichment) -- node[anchor=east]{Tails}(sinkllw);
        \draw [arrow] (enrichment2) -- node[anchor=west]{Tails}(sinkllw);
        \draw [arrow] (fabrication) -- node[anchor=east]{Fresh UOX}(reactor);
        \draw [arrow] (fabrication2) -- node[anchor=west]{TRSIO fuel}(adv_reactor);
        \draw [arrow] (reactor) -- node[anchor=east]{Spent UOX}(wetstorage);
        \draw [arrow] (wetstorage) -- node[anchor=east]{Cool Spent UOX}(drystorage);
        \draw [arrow] (drystorage) -- node[anchor=east]{Casked Spent UOX}(sinkhlw);
        \draw [arrow] (adv_reactor) -- node[anchor=west]{Spent TRISO Fuel}(sinkhlw);

        \end{tikzpicture}
    \caption{Fuel cycle facilities and material flow between facilities. Facilities in 
    red are deployed in the transition scenarios.}
    \label{fig:fuel_cycle}
\end{figure}

Other fuel cycle scenarios simulated include the transition to a future with 
substantial \gls{HALEU} 
reactor deployment under two assumptions about growth in nuclear energy 
demand: one assuming no growth and the other assuming 1\% annual growth. 
The transition to the new reactor type starts in 2025. 
These scenarios used the same  
non-reactor facilities as in the simulation of the current 
U.S. fuel cycle. \gls{HALEU} fuel reactors 
considered in this work include the \gls{USNC} \gls{MMR} \textsuperscript{TM}
\cite{mitchell_usnc_2020} and the X-Energy Xe-100 \textsuperscript{TM} 
Reactor \cite{harlan_x-energy_2018}\cite{hussain_advances_2018}. Both of 
these reactors are designed 
to use \gls{HALEU} fuel in the form of \gls{TRISO} fuel particles. Table 
\ref{tab:reactor_summary} summarizes the key design characteristics of these 
two reactors.

\begin{table}[ht]
    \caption{Mico-reactor design specifications}
    \label{tab:reactor_summary}
    \begin{tabular}{p{2.5cm}p{2.25cm}p{2.5cm}}
        \hline
        Design Criteria & \gls{USNC}\gls{MMR}\textsuperscript{TM} & 
            X-Energy Xe-100\textsuperscript{TM} \\\hline
        Reactor type & Modular HTGR & Modular HTGR \\
        Power Output (MWth) & 15 & 200 \\
        Enrichment (\% $^{235}U$) & 13 & 15.5 \\
        Cycle Length (years) & 20 & online refuel\\
        Fuel form & \gls{TRISO} compacts & \gls{TRISO} pebbles\\
        Reactor Lifetime & 20 years & 60 years \\
        Coolant & He & He \\
        \hline
    \end{tabular}
\end{table}
    
We selected these two reactors because they have a high 
likelihood of being deployed and because published information is 
available about their designs. Comparing the transitional \gls{HALEU} 
demands of these reactors will also inform the fuel cycle support 
implications of deploying reactors with long cycle 
times versus those utilizing online refueling. 

Each of the transition scenarios will be simulated using a single type of 
advanced reactor, resulting in four additional fuel cycle scenarios and five 
scenarios in total. Each of the scenarios will be analyzed using Cymetric
\cite{scopatz_cymetric_2015} to determine the resource requirements of the 
scenario. 