\section{Methodology}

Each fuel cycle scenario is simulated using \Cyclus, an 
agent-based fuel cycle simulator \cite{huff_fundamental_2016}. 
The agent-based architecture of \Cyclus allows for the simulation to treat
each facility discretely, including how they are deployed and 
decommissioned \cite{huff_fundamental_2016}. Each of the facility agents 
deployed in the simulation are from the \Cycamore archetype library. 

The first scenario simulated is the current fleet of U.S. reactors. This 
simulation begins in January 1965 and lasts through December 2050 (85 
years in total). Information about individaul reactors was obtained 
from the 2020 Year-end Reactor Status Report of the \gls{IAEA} \gls{PRIS} 
database \cite{noauthor_power_1989}. 
This database provides reactor type, rated power level, start-up, and 
shutdown dates for each of the commercial reactors in the U.S. reactor
fleet
that are still operational (as of December 2020). The reactors are assumed 
to operate for 60 years after the 
initial start date if they have not already been shutdown. Only reactors
with a power level above 400 MWe were 
used in the simulation to remove reactors that are assumed to be research 
reactors. Generic reactor core masses were obtained from 
\cite{todreas_nuclear_2012} and \cite{cacuci_handbook_2010}. 

Other fuel cycle facilities in the simulation include a uranium mine (to 
represent the Cameco Smith Ranch-Highland Mine), a uranium mill (to 
represent White Mesa Uranium Mill), a conversion facility (to represent 
the Honeywell Uranium Conversion Facility), an enrichment plant (to represent 
the Portsmouth Gaseous Diffusion Plant), a fuel fabrication facility (to 
represent the Westinghouse Fuel Fabrication Facility), and various waste and 
spent fuel storange facilities (to represent Yucca Mountain). While there
are other fuel cycle facilities in the actual U.S. fuel cycle, this 
modeling provides a total of the resources required 
by each non-reactor step in the nuclear fuel cycle and simplifies the 
facilities that are not expected to be deployed or decommissioned often 
during the specified time frame. 

Other scenarios simulated include the transition to \gls{HALEU} fuel 
reactors. These scenarios are based on two models: no growth in demand 
and an annual 1\% growth in demand. These scenarios will use the same 
non-reactor facilities that are used in the simulation for the current 
U.S. fuel cycle. \gls{HALEU} fuel reactors 
considered in this work include the \gls{USNC} \gls{MMR} \textsuperscript{TM}
\cite{mitchell_usnc_2020} and the X-Energy Xe-100 \textsuperscript{TM} 
Reactor \cite{harlan_x-energy_2018}\cite{hussain_advances_2018}. Both of 
these reactors are designed 
to use \gls{HALEU} fuel in the form of \gls{TRISO} fuel pebbles. Table 
\ref{tab:reactor_summary} summarizes the design of these two reactors.

\begin{table}
    \caption{Mico-reactor design specifications}
    \label{tab:reactor_summary}
    \begin{tabular}{|p{2.5cm}|p{2.25cm}|p{2.5cm}|}
        \hline
        Design Criteria & \gls{USNC}\gls{MMR}\textsuperscript{TM} & 
            X-Energy Xe-100\textsuperscript{TM} \\\hline
        Reactor type & Modular HTGR & Modular HTGR \\
        Power Output (MWth) & 15 & 200 \\
        Enrichment (\% $^{235}U$) & 13 & 15.5 \\
        Cycle Length (years) & 20 & online refuel\\
        Fuel form & \gls{TRISO} & \gls{TRISO} \\
        Reactor Lifetime & 20 years & 60 years \\
        Coolant & He & He \\
        \hline
    \end{tabular}
\end{table}
    
We selected these two reactors because they have a high 
likelihood of being deployed and open source information is available about 
their designs. These reactors will also allow a comparison in the 
deployment of advanced reactors with a long cycle time and those that 
utilize online refueling. 

Each of the transition scenarios will be simulated using a single type of 
micro-reactor, resulting in four additional fuel cycle scenarios and five 
scenarios in total. Each of the scenarios will be analyzed using Cymetric
\cite{scopatz_cymetric_2015} to determine the resource requirements of the 
scenario. 