\section{Introduction}
The \gls{LWR} that are currently employed for commercial power in the 
United States use similar fuel forms at \gls{LEU} enrichment levels 
less than 5\%. New reactor designs, such as the \gls{USNC} \gls{MMR}, will 
use \gls{LEU} fuel enriched between 5-20\%, often referred to as \gls{HALEU}.
Changing the enrichment level of commercial reactor fuel will require 
an increase in multiple front-end resources to produce the same mass of fuel. 

To meet the demand for \gls{HALEU} fuel for advanced reactors, the U.S. 
\gls{DOE} has proposed two methods to achieve fuel at the 
required enrichment level: recovery and downblending of \gls{HEU} fuel 
from EBR-II and other \gls{DOE} owned \gls{HEU} fuel or enriching natural 
uranium to the desired level \cite{griffith_overview_2020}. Each of these 
methods to produce \gls{HALEU} fuel has its limitations. Downblending
\gls{HEU} fuel is limited by the existing physical supply of \gls{HEU}
fuel and downblending capacity. Enrichment of natural uranium is limited by
the centrifuge capacity in terms of \gls{SWU} and throughput.

This works simulates multiple fuel cycle scenarios to compare methods 
to meeting \gls{HALEU} fuel demand for advanced reactors. Each simulated 
scenario will be investigated for the demand of enriched uranium by the 
reactors and the resources required to meet this demand. 