\section{Introduction}
The \glspl{LWR} that are currently employed for commercial power in the 
United States use \gls{UOX} fuel at enrichment levels 
less than 5\%. New reactor designs, such as the \gls{USNC} \gls{MMR}, will 
use \gls{LEU} fuel enriched between 5-20\%, often referred to as \gls{HALEU}.
Deploying commercial reactors fueled by a different enrichment level 
will require 
changes in multiple front-end resources, such as enrichment 
feed material and \gls{SWU} capacity, to produce the same mass of fuel. 

To meet the \gls{HALEU} fuel demand for advanced reactors, the U.S. 
\gls{DOE} has proposed two methods to achieve fuel at the 
required enrichment level: recovery and downblending of \gls{HEU} fuel 
from EBR-II and other \gls{DOE} \gls{HEU} stockpiles or enriching natural 
uranium to the desired level \cite{griffith_overview_2020}. Both of these 
methods to produce \gls{HALEU} fuel have limitations. Domestic downblending 
capabilities and the existing physical supply of \gls{HEU} both ultimately 
limit the potential for generating \gls{HALEU} by downblending. Meanwhile, 
available \gls{SWU} capacity and throughput of domestic centrifuge 
facilities together limit the potential to generate \gls{HALEU} by 
enriching natural uranium. 

This work simulates multiple fuel cycle scenarios to investigate 
material demands for the transition to \gls{HALEU} fueled advanced 
reactors. Each simulated 
scenario will investigate the demand of enriched uranium by the 
reactors and the resources required to meet this demand. 